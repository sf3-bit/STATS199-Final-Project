% Options for packages loaded elsewhere
\PassOptionsToPackage{unicode}{hyperref}
\PassOptionsToPackage{hyphens}{url}
%
\documentclass[
]{article}
\usepackage{amsmath,amssymb}
\usepackage{iftex}
\ifPDFTeX
  \usepackage[T1]{fontenc}
  \usepackage[utf8]{inputenc}
  \usepackage{textcomp} % provide euro and other symbols
\else % if luatex or xetex
  \usepackage{unicode-math} % this also loads fontspec
  \defaultfontfeatures{Scale=MatchLowercase}
  \defaultfontfeatures[\rmfamily]{Ligatures=TeX,Scale=1}
\fi
\usepackage{lmodern}
\ifPDFTeX\else
  % xetex/luatex font selection
\fi
% Use upquote if available, for straight quotes in verbatim environments
\IfFileExists{upquote.sty}{\usepackage{upquote}}{}
\IfFileExists{microtype.sty}{% use microtype if available
  \usepackage[]{microtype}
  \UseMicrotypeSet[protrusion]{basicmath} % disable protrusion for tt fonts
}{}
\makeatletter
\@ifundefined{KOMAClassName}{% if non-KOMA class
  \IfFileExists{parskip.sty}{%
    \usepackage{parskip}
  }{% else
    \setlength{\parindent}{0pt}
    \setlength{\parskip}{6pt plus 2pt minus 1pt}}
}{% if KOMA class
  \KOMAoptions{parskip=half}}
\makeatother
\usepackage{xcolor}
\usepackage[margin=1in]{geometry}
\usepackage{graphicx}
\makeatletter
\def\maxwidth{\ifdim\Gin@nat@width>\linewidth\linewidth\else\Gin@nat@width\fi}
\def\maxheight{\ifdim\Gin@nat@height>\textheight\textheight\else\Gin@nat@height\fi}
\makeatother
% Scale images if necessary, so that they will not overflow the page
% margins by default, and it is still possible to overwrite the defaults
% using explicit options in \includegraphics[width, height, ...]{}
\setkeys{Gin}{width=\maxwidth,height=\maxheight,keepaspectratio}
% Set default figure placement to htbp
\makeatletter
\def\fps@figure{htbp}
\makeatother
\setlength{\emergencystretch}{3em} % prevent overfull lines
\providecommand{\tightlist}{%
  \setlength{\itemsep}{0pt}\setlength{\parskip}{0pt}}
\setcounter{secnumdepth}{5}
\usepackage{hyperref}
\usepackage{caption}
\usepackage{xcolor}
\usepackage{float}
\ifLuaTeX
  \usepackage{selnolig}  % disable illegal ligatures
\fi
\usepackage{bookmark}
\IfFileExists{xurl.sty}{\usepackage{xurl}}{} % add URL line breaks if available
\urlstyle{same}
\hypersetup{
  pdftitle={199 Report},
  pdfauthor={Stephen Fujiwara},
  hidelinks,
  pdfcreator={LaTeX via pandoc}}

\title{199 Report}
\author{Stephen Fujiwara}
\date{March 14, 2025}

\begin{document}
\maketitle

\section{Introduction}\label{introduction}

In this analysis, we explore the extent to which well-known market
effects from traditional financial markets, such as seasonality,
momentum, mean-reversion, carry, etc. are present in the cryptocurrency
markets. These phenomena have been well-documented in the more
traditional financial markets, such as equities and fixed income, where
they are leveraged in various trading strategies. However, the unique
idiosyncrasies of crypto markets that separate them from traditional
markets - such as their relatively young infrastructure,
decentralization, volatility, the risk-appetite of customers - suggest
that these effects may be exacerbated, behave differently, or perhaps
fail to manifest altogether.

Specifically, in this analysis, we investigate seasonality effects in
cryptocurrency markets, focusing on patterns based on the day of the
week and patterns based on the day of the month. Additionally, we
examine the presence of trend effects and their potential role in price
movements. Moreover, on top of conducting these investigations across an
aggregate of cryptocurrencies, we also repeated each of these
investigations specifically with Bitcoin. To conduct this analysis, we
collected daily price data from the past 5 years from the top 50
cryptocurrencies ranked by market capitalization using CoinGecko's API.
CoinGecko is a provider of cryptocurrency data that offers detailed
information across a wide range of coins in the cryptocurrency industry.
The goal is to assess whether these phenomena hold relevance in the
crypto space, and with these phenomena in mind, to identify unique
patterns that could inform quantitative trading strategies in this asset
class.

\section{Data Overview, Collection,
Pre-Processing}\label{data-overview-collection-pre-processing}

The data consists of a single dataset. This dataset contains the daily
prices, beginning on the 1st of January, 2020 to the 1st January, 2025,
from the top 50 cryptocurrencies ranked by market capitalization on the
date of collection.

\subsection{Data Collection}\label{data-collection}

The data for this analysis was collected via the CoinGecko API, which
provides comprehensive, granular data on the cryptocurrency market.
Initially, a request was made to retrieve the list of the top
cryptocurrencies ranked by market capitalization. To ensure the
relevance of the dataset and reduce redundancy, stablecoins -
cryptocurrencies designed to maintain a stable value, and wrapped tokens
- representations of other cryptocurrencies for trading at different
venues - were filtered out. Following this, separate requests were made
for each of the top 50 coins to obtain their daily closing prices over
the past five years, alongside it's timestamp. The resulting time-series
data for each asset was then merged using an outer join on the
timestamp, to construct a single, comprehensive dataset suitable for
analysis.

\subsection{Data Pre-Processing}\label{data-pre-processing}

To facilitate the analysis of market effects, the daily closing price
data for each cryptocurrency is transformed into daily log returns.
Logarithmic returns, defined as \(log(\frac{P_t}{P_{t-1}})\), are
calculated for each timestamp, where \(P_t\) represents the closing
price of the asset at time \(t\). This transformation is commonly used
in analysis of financial markets because it produces returns that are
time-additive, making it easier to model and plot. By using log returns,
the analysis accounts for the effects of compounding, enabling a robust
assessment of market dynamics.

From the single, comprehensive dataset, we then construct a dataset
comprising of the log returns of the top 50 cryptocurrencies in
aggregate, where for each day, we take the average of the individuals
log returns of each cryptocurrency and average them to a single value.
This effectively serves as the dataset of the log returns of an
equal-weighted average of the top 50 coins, providing a broader measure
of market performance.

Additionally, we create a separate dataset of Bitcoin's returns alone,
isolating its returns from to evaluate it independently from the broader
market. These smaller sets of data serve as the foundation for the
subsequent seasonality and trend analysis.

\section{Seasonality Analysis (Day of the
Week)}\label{seasonality-analysis-day-of-the-week}

\subsection{Aggregate (Equal-Weighted Top 50
Index)}\label{aggregate-equal-weighted-top-50-index}

\subsubsection{Visual Analysis}\label{visual-analysis}

To assess whether cryptocurrency demonstrate systemic patterns across
the days of the week, we analyze the aggregate time series by grouping
daily log returns according to the day of the week. Initially, we
compute the average log return for each day of the week and plot them
alongside each other. Figure \textcolor{blue}{\ref{fig:1}} reveals a
pronounced weekend effect, where there are strongly positive average
returns on Saturday and Sunday, while the rest of the week appears
noisy, with no clear pattern.

\begin{figure}
    \centering
    \includegraphics[width=0.5\textwidth]{figure1.png}
    \caption{Mean Return of The Aggregate According to the Day of the Week}
    \label{fig:1}
\end{figure}

To mitigate the influence of outliers, we complement this analysis by
doing the same with the median log return of each day of the week. The
median, here, can serve as a robust measure as it reduces the impact of
very extreme values that could distort the general pattern. Figure
\textcolor{blue}{\ref{fig:2}} confirms the weekend effect, showing that
while all days of the week exhibit modestly positive returns, the
weekend stands out with higher returns relative to the rest of the week,
although not by a pronounced margin.

\begin{figure}
    \centering
    \includegraphics[width=0.5\textwidth]{figure2.png}
    \caption{Median Return of The Aggregate According to the Day of the Week}
    \label{fig:2}
\end{figure}

Given these simple visual tests, we fail to reject the idea that the
returns according to the day of the week are purely random. Therefore,
we continue on and construct a simple metric to quantify the weekend
effect in a more explicit manner. The metric goes as such: for each
week, we compute the difference between the sum of the returns of the
weekend and the sum of the returns of the rest of the week. We then
visualize this metric as a bar plot in Figure
\textcolor{blue}{\ref{fig:3}}. There appear to be slightly more weeks
where the metric is positive than it is negative, with the magnitudes
distributed nearly symmetrically, but overall the results seem
inconclusive.

\begin{figure}
    \centering
    \includegraphics[width=0.5\textwidth]{figure3.png}
    \caption{Bar Chart of the Weekly Metric of The Aggregate}
    \label{fig:3}
\end{figure}

To further evaluate this effect, we compute and plot the cumulative sum
of this weekly metric over time, effectively simulating the performance
of a strategy that seeks to exploit the weekend anomaly. Figure
\textcolor{blue}{\ref{fig:4}} suggests that an account following the
strategy from 2020 to 2025 would have approximately doubled in value in
log return terms, translating to a 539\% increase in normal return
terms; however, the path is highly volatile and noisy.

\begin{figure}
    \centering
    \includegraphics[width=0.5\textwidth]{figure4.png}
    \caption{Cumulative Log Returns of a Simulated Day of the Week Seasonality Strategy on an Equal-Weighted Index of the Top 50 Coins ex. Transaction Costs, Slippage, etc.}
    \label{fig:4}
\end{figure}

\subsubsection{Statistical Testing}\label{statistical-testing}

Since our visual tests didn't outright reject the presence of the
weekend effect, we proceed to formal statistical tests. To formally test
whether the observed weekend effect is statistically significant, we
conducted two non-parametric hypothesis tests: the Kruskal-Willis Test
and a permutation test.

The Kruskal-Willis test, a generalization of ANOVA, assesses whether
samples originate from the same distribution without assuming normality.
This test is particularly useful, since cryptocurrency returns may
exhibit heavy tails. The Kruskal Willis test returns a p-value of 0.26,
which at a significance level of 0.05 leads us to fail to reject the
null, suggesting no strong evidence against purely random returns.

We then perform a permutation test, shuffling the day of the week
associated with the returns to produce a null distribution for the test
statistics, defined as the difference in mean returns between the
weekend and the rest of the week. Doing so allows us to investigate
whether the observed weekend premium is likely to have been due to
chance. The permutation test yields a p-value of 0.08, which, while
lower than that of the Kruskal-Willis test, remains above the
conventional significance level of 0.05, leading us to fail to reject
the null hypothesis.

While the weekend effects appears to be visually suggestive, our
statistical tests do not provide sufficient evidence to conclude that it
is a persistent, exploitable effect, and if it is, it is an effect
subject to significant variability in outcome, and may take many years
to realize it's expected returns.

\subsection{Bitcoin}\label{bitcoin}

\subsubsection{Visual Analysis}\label{visual-analysis-1}

As in the previous analysis done with the aggregate, we examine the
effect by grouping the returns of Bitcoin by the day of the week, and
computing both the mean and median for each day. As Figure
\textcolor{blue}{\ref{fig:5}} and Figure \textcolor{blue}{\ref{fig:6}}
show, unlike the broader market's weekend effect, Bitcoin exhibits a
different pattern: Mondays and Fridays show significantly lower returns
relative to the rest of the week, where Fridays are strongly negative.

\begin{figure}[H]
    \centering
    \begin{minipage}{0.45\textwidth}
        \centering
        \includegraphics[width=\textwidth]{figure5.png}
        \caption{Mean Return of Bitcoin According to the Day of the Week}
        \label{fig:5}
    \end{minipage}
    \hfill
    \begin{minipage}{0.45\textwidth}
        \centering
        \includegraphics[width=\textwidth]{figure6.png}
        \caption{Median Return of Bitcoin According to the Day of the Week}
        \label{fig:6}
    \end{minipage}
\end{figure}

Given this observed idiosyncrasy, we construct a simple metric that
captures the difference in the sum of the returns of the proposed
stronger days and the sum of the returns of the weaker days. Much like
in the previous analysis, the metric is slightly frequently positive
than negative, and the magnitudes are about the same, but mostly
inconclusive, as Figure \textcolor{blue}{\ref{fig:7}} demonstrates. When
plotted cumulatively, Figure \textcolor{blue}{\ref{fig:8}} shows that
metric suggests that an account following the strategy would have
tripled in log return terms from 2020 to 2025, equating to a 1,908\%
return, although with some variability.

\begin{figure}[H]
    \centering
    \begin{minipage}{0.45\textwidth}
        \centering
        \includegraphics[width=\textwidth]{figure7.png}
        \caption{Bar Chart of the Weekly Metric of Bitcoin}
        \label{fig:7}
    \end{minipage}
    \hfill
    \begin{minipage}{0.45\textwidth}
        \centering
        \includegraphics[width=\textwidth]{figure8.png}
        \caption{Cumulative Log Returns of a Simulated Day of the Week Seasonality Strategy on Bitcoin ex. Transaction Costs, Slippage, etc.}
        \label{fig:8}
    \end{minipage}
\end{figure}

\subsubsection{Statistical Testing}\label{statistical-testing-1}

Performing the same tests as in the previous analysis, the
Kruskal-Willis test produces a p-value of 0.62, and the permutation test
returns a p-value of 0.09 - both failing to reject the null hypothesis
at the significance level of 0.05. While it seems that behavior of the
price movements of Bitcoin on a day of the week basis may be different
than that of the broader market, our findings do not provide strong
enough evidence to reject the proposition that it may simply be due to
noise.

\section{Seasonality Analysis (Day of the
Month)}\label{seasonality-analysis-day-of-the-month}

\subsection{Aggregate (Equal-Weighted Top 50
Index)}\label{aggregate-equal-weighted-top-50-index-1}

\subsubsection{Visual Analysis}\label{visual-analysis-2}

To examine potential seasonality effects based on parts of the month, we
begin by grouping daily log returns according to their calendar day and
computing the mean return for each day. As Figure
\textcolor{blue}{\ref{fig:9}} indicates, returns in the first and last
week of the month were consistently positive, while the rest of the
month exhibits considerable noise. We also compute the median returns
grouped by day of the month as well, and as Figure
\textcolor{blue}{\ref{fig:10}} reveals, interestingly, the median
returns are mostly positive across the board. While it shows no clear
pattern, it demonstrates an interesting characteristic of the daily
returns, which is that they are skewed with a heavy left tail.

\begin{figure}
    \centering
    \begin{minipage}{0.45\textwidth}
        \centering
        \includegraphics[width=\textwidth]{figure5.png}
        \caption{Mean Return of The Aggregate According to the Day of the Month}
        \label{fig:9}
    \end{minipage}
    \hfill
    \begin{minipage}{0.45\textwidth}
        \centering
        \includegraphics[width=\textwidth]{figure6.png}
        \caption{Median Return of The Aggregate According to the Day of the Month}
        \label{fig:10}
    \end{minipage}
\end{figure}

To further investigate, we repeat the computation of the mean return
according to the day of the month, for each year of the past five years.
This breakdown, demonstrated in Figure \textcolor{blue}{\ref{fig:11}},
reveals a persistent effect: returns in the final five days of the month
are consistently positive across all years, suggesting a month-end
anomaly.

\begin{figure}[H]
    \centering
    \includegraphics[width=0.75\textwidth]{figure11.png}
    \caption{Breakdown by Year of the Mean Returns of The Aggregate According to the Day of the Month}
    \label{fig:11}
\end{figure}

To quantify this effect, we define a simple monthly metric, computed as
the sum of the log returns over the last five days of each month, and
plot it as a bar plot. Figure \textcolor{blue}{\ref{fig:12}} shows this
metric is slightly more frequently positive than negative, and the
magnitudes are roughly balanced. As Figure
\textcolor{blue}{\ref{fig:13}} indicates, a cumulative plot of this
metric over time suggests that a strategy exclusively exploiting this
pattern would have a cumulative log return of 0.9, translating to 146\%
return in regular return terms (excluding transaction costs, slippage,
etc.).

\begin{figure}[H]
    \centering
    \begin{minipage}{0.45\textwidth}
        \centering
        \includegraphics[width=\textwidth]{figure12.png}
        \caption{Bar Chart of the Monthly Metric of The Aggregate}
        \label{fig:12}
    \end{minipage}
    \hfill
    \begin{minipage}{0.45\textwidth}
        \centering
        \includegraphics[width=\textwidth]{figure13.png}
        \caption{Cumulative Log Return of a Simulated Day of the Month Seasonality Strategy on an Equal-Weighted Index of the Top 50 Coins ex. Transaction Costs, Slippage, etc.}
        \label{fig:13}
    \end{minipage}
\end{figure}

\subsubsection{Statistical Testing}\label{statistical-testing-2}

Moving beyond our visual analysis, which suggests an end-of-month
effect, we move to statistical tests to formally assess the presence of
this effect.

A Kruskal-Willis test, which evaluates whether returns vary
systematically across different days of the month, produces a p-value of
0.31, suggesting no statistically significant differences. In addition,
a permutation test, where we permute the day of the month corresponding
to each return observation for each month, and compare against the
observed test statistic of the difference in mean returns between the
final days of the month and the rest of the month. This test yields a
p-value of 0.43, failing to reject the null hypothesis, the assumption
that returns on different days of the month are drawn from the same
distribution, and that any observed patterns are likely due to chance.

Although visually suggestive, there is not strong enough evidence to
reject the assumption that this perceived pattern has arisen purely due
to chance. Or, if the effect were to exist, it could simply be too weak
and too noisy to detect given our sample size.

\subsection{Bitcoin}\label{bitcoin-1}

\subsubsection{Visual Analysis}\label{visual-analysis-3}

As in our previous analysis of seasonality based on the day of the month
with the equal-weighted aggregate, we conduct an analysis focusing
solely on Bitcoin. We begin by grouping the log returns by the month and
computing the mean. As Figure \textcolor{blue}{\ref{fig:14}} indicates,
the returns at the end of the monthly are strongly positive, with the
rest of the month exhibiting significant noise. As for the median,
Figure \textcolor{blue}{\ref{fig:15}} demonstrates results that align
very closely with that of the mean grouped by day of the month.

\begin{figure}[H]
    \centering
    \begin{minipage}{0.45\textwidth}
        \centering
        \includegraphics[width=\textwidth]{figure14.png}
        \caption{Mean Return of Bitcoin According to the Day of the Month}
        \label{fig:14}
    \end{minipage}
    \hfill
    \begin{minipage}{0.45\textwidth}
        \centering
        \includegraphics[width=\textwidth]{figure15.png}
        \caption{Median Return of Bitcoin According to the Day of the Month}
        \label{fig:15}
    \end{minipage}
\end{figure}

To determine it's persistence more granularly, we repeat the computation
of the mean return for each day of the month for each year of the past
five years. As Figure \textcolor{blue}{\ref{fig:16}} indicates, across
all years, the last five days of the month exhibit consistently positive
returns in each year, with the rest of the month showing no clear
pattern.

\begin{figure}[H]
    \centering
    \includegraphics[width=0.75\textwidth]{figure16.png}
    \caption{Breakdown by Year of the Mean Returns of Bitcoin According to the Day of the Month}
    \label{fig:16}
\end{figure}

Further analyzing this effect, we compute the metric utilized in the
previous analysis, and plot it. According to Figure
\textcolor{blue}{\ref{fig:17}}, the metric appears to be more frequently
positive than negative, with similar magnitudes in both directions. A
cumulative plot of this metric shows that a strategy based on exploiting
this end-of-month effect would have achieved a cumulative log return of
1, or a 172\% return over the period studied, as indicated in Figure
\textcolor{blue}{\ref{fig:18}}.

\begin{figure}[H]
    \centering
    \begin{minipage}{0.45\textwidth}
        \centering
        \includegraphics[width=\textwidth]{figure17.png}
        \caption{Bar Chart of the Monthly Metric of Bitcoin}
        \label{fig:17}
    \end{minipage}
    \hfill
    \begin{minipage}{0.45\textwidth}
        \centering
        \includegraphics[width=\textwidth]{figure18.png}
        \caption{Cumulative Log Return of a Simulated Day of the Month Seasonality Strategy on Bitcoin ex. Transaction Costs, Slippage, etc.}
        \label{fig:18}
    \end{minipage}
\end{figure}

\subsubsection{Statistical Testing}\label{statistical-testing-3}

Testing the statistical validity of this anomaly, a Kruskal-Willis test
yields a p-value of 0.15, while a permutation test - structured the same
way as in the previous analysis - yields a p-value of 0.23. As in the
aggregate case, these results cause us to fail to reject the null
hypothesis of non-seasonality.

\section{Trend (Time-Series Momentum)
Analysis}\label{trend-time-series-momentum-analysis}

\subsection{Aggregate (Equal - Weighted Top 50
Index)}\label{aggregate-equal---weighted-top-50-index}

\subsubsection{Visual Analysis}\label{visual-analysis-4}

To assess whether trend effects are present in the aggregate, we examine
the cumulative log return over the past five years. Figure
\textcolor{blue}{\ref{fig:19}} suggests periods of trending behavior,
but further analysis is required.

\begin{figure}[H]
    \centering
    \includegraphics[width=0.75\textwidth]{figure19.png}
    \caption{Cumulative Log Returns of an Equal-Weighted Index of the Top 50 Coins from 2020 to 2025}
    \label{fig:19}
\end{figure}

We then analyze the relationship between past and future returns by
computing the average log return for each period in the dataset, where a
period is defined as a series of days, and plotting it against the
average return of the following period, along with a fitted regression
line. Doing this for different period lengths (e.g.~1 day, 3 days, 5
days, etc.). If returns exhibit trending behavior, we would expect a
positive slope. As Figure \textcolor{blue}{\ref{fig:20}} indicates,
there is a slightly negative slope for a 1-day period, a slightly
positive slope for a 3-day period, and beginning with a 5-day period the
slope increases up to when the period length reaches 20 days, before
declining once again. This pattern visually suggests that returns may
exhibit trending behavior over intermediate time horizons, though the
effect weakens at longer horizons.

\begin{figure}[H]
    \centering
    \includegraphics[width=0.75\textwidth]{figure20.png}
    \caption{Plot of Previous Period's Mean Log Return against Next Period's Mean Log Return for Periods of lengths: 1 day, 3 days, 5 days, etc.}
    \label{fig:20}
\end{figure}

To further investigate, we compute the autocorrelation function (ACF) of
log returns up to a lag of 30 days. Figure
\textcolor{blue}{\ref{fig:21}} shows that the ACF immediately drops
close to zero after the first lag, indicating little persistence in
returns. However, this result on it's own is not conclusive, as the ACF
measures correlations at individual lags, but does not account for
dependencies that may emerge jointly across multiple periods.

\begin{figure}[H]
    \centering
    \includegraphics[width=0.75\textwidth]{figure21.png}
    \caption{ACF (Autocorrelation Function) of The Aggregate}
    \label{fig:21}
\end{figure}

\subsubsection{Statistical Testing}\label{statistical-testing-4}

To statistically test for the presence of a trend effect, we conduct the
Ljung-Box test on the log returns up to a lag of 30 days. This test
examines whether the autocorrelations of a time series are jointly equal
to zero. The results for this test yield p-values below starting at
0.007 and decreasing to a minimum of 0.000169 at a lag of 25, before
rising again. This suggests some degree of autocorrelation up to larger
lags.

We also fit an AR (AutoRegressive) model with up to 30 lags. An AR model
models how past returns can predict future returns by expressing the
return at time \(t\), \(r_t\), as a linear combination of previous
return values \(r_{t-1}, r_{t-2}\) etc. The p-values of the estimated
coefficients of this model vary widely, ranging from 0.01 to 0.98 with
no discernible pattern - some lags exhibit statistical significance,
while others do not, with no apparent relationship with respect to lag
length. This lack of structure does not provide sufficient evidence to
indicate the presence of a trend effect.

While the analysis hints at the presence of trending behavior, the
statistical evidence is not sufficient enough to support it. If trends
exist in the broader market, they are weak, and dominated by noise.

\subsection{Bitcoin}\label{bitcoin-2}

\subsubsection{Visual Analysis}\label{visual-analysis-5}

To examine whether Bitcoin exhibits trend effects, we conduct an
analysis identical to that applied to the aggregate index. Plotting the
cumulative log returns of Bitcoin over the past five years, visual
inspection suggests the presence of trends, as Figure
\textcolor{blue}{\ref{fig:22}} indicates.

\begin{figure}[H]
    \centering
    \includegraphics[width=0.75\textwidth]{figure22.png}
    \caption{Cumulative Log Returns of Bitcoin from 2020 to 2025}
    \label{fig:22}
\end{figure}

Plotting the average log return for each period against the average log
return of the next period and fitting a regression line, for periods up
to 50 days, we see results similar to that of the aggregate. Figure
\textcolor{blue}{\ref{fig:23}} shows that a 1-day period the slope is
slightly negative, but as period lengths are increased, the slope
increases just as well, peaking at a period length of 30 days, before
declining again. This pattern suggests that returns exhibit trending
behavior over intermediate timeframes, but weakens at much shorter or
much longer timeframes, much like the aggregate.

\begin{figure}[H]
    \centering
    \includegraphics[width=0.75\textwidth]{figure23.png}
    \caption{Plot of Previous Period's Mean Log Return against Next Period's Mean Log Return for Periods of lengths: 1 day, 3 days, 5 days, etc.}
    \label{fig:23}
\end{figure}

Examining further, we compute and plot the ACF (Figure
\textcolor{blue}{\ref{fig:24}}). Similar to the aggregate, the
autocorrelation drops to zero immediately after the first lag,
suggesting little to no persistence in daily returns on the time scale
we are investigating.

\begin{figure}[H]
    \centering
    \includegraphics[width=0.75\textwidth]{figure24.png}
    \caption{ACF (Autocorrelation Function) of Bitcoin}
    \label{fig:24}
\end{figure}

\subsubsection{Statistical Testing}\label{statistical-testing-5}

For a statistical test of trending behavior, we conduct a Ljung-Box test
up to 50 lags. The results show a p-value of 0.058 at lag 1, 0.029 at
lag 2, and 0.064 at lag 3. Beyond this, the p-values remain below the
conventional significance level of 0.05, declining until lag 47, where
they begin to increase noticably again.

We further examine the presence of trend effects by fitting an AR model
to Bitcoin's log returns. The resulting p-values range from as low as
0.013 to as high as 0.936, without any discernible pattern - some lags
may be statistically significant, but overall the significance does not
systematically increase or decrease with lag length. This erratic
distribution of p-values, indicate that any dependencies are
inconsistent, and the returns do not follow a clear trend-based
structure.

While certain analyses may suggest the possibility of trending behavior
over intermediate timeframes, the statistical evidence remains weak.
These results align with those observed for the aggregate index, where
if trend effects exist, they may not be strong enough to be exploited.

\section{Discussion and Suggestions}\label{discussion-and-suggestions}

Given the characteristics of cryptocurrency markets - such as their
relative nascence, lower regulatory oversight, and high retail
participation - one might expect these markets exhibit inefficiencies
similar to those well-known in traditional financial markets, and
potentially in a more pronounced manner. However, our analysis suggests
that this may not be the case, at least for the market anomalies we
specifically investigated.

We examined seasonality effects based on both the day of the week as
well as the day of the month, resulting in not sufficiently strong
evidence to reject that any pattern may have been due to chance.
Similarly, when investigating the presence of trend effects, we found
mixed results. While visual inspection suggested some degree of trending
behavior, statistical tests produced inconclusive results.

It is important, however, for us to acknowledge the limitations of our
study. First, we have a relatively small sample size of five years worth
of daily price data. This data may reflect recent market dynamics, but
it does not allow us to investigate inefficiencies on a much longer-term
basis throughout cryptocurrency history. In addition, we focused our
investigation on a specific set of well-documented market effects. There
are numerous other inefficiencies that we did not explore, such as
momentum effects at different time scales, cross-sectional effects such
as size and low-volatility, or even inefficiencies unique to crypto -
such as the structural inefficiencies related to funding rates and
perpetual swap dynamics.

Future research could extend this analysis in multiple ways. Expanding
the dataset length to incorporate a more extensive history could provide
a much stronger analysis of the inefficiencies we investigated, with a
much larger sample size. Incorporating intra-day data might also reveal
short-term inefficiencies that were masked on a daily timescale. In
addition, expanding the scope of analysis to alternative inefficiencies
- such as those related to cross-sectional factors, exchange-specific
price deviations, or the impact of large liquidations, etc. - could
provide a more comprehensive assessment of inefficiencies in the
cryptocurrency markets and their similarities or lack thereof with the
inefficiencies that exist in traditional financial markets.

\section{Conclusion}\label{conclusion}

Overall, while our findings suggest that certain well-known market
inefficiencies may not be present in cryptocurrency markets as one might
expect, this does not imply that inefficiencies are absent altogether.
Rather, it suggests for further research to identify and understand the
unique dynamics at play in these markets.

\end{document}
